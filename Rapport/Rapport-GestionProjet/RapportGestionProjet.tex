\documentclass{report}
\usepackage[utf8]{inputenc}
\usepackage{xcolor}
\usepackage{sectsty} %%Pour changer la couleur de la police 

\title{Rapport}
\author{}
\date{}


\renewcommand{\contentsname}{Table des matières}
\renewcommand{\chaptername}{Chapitre}
\renewcommand{\appendixname}{Annexe}
\renewcommand{\bibname}{Bibliographie}

\definecolor{bleurapport}{HTML}{00AAD4}

\chapterfont{\color{bleurapport}}
\sectionfont{\color{bleurapport}}
\subsectionfont{\color{bleurapport}}

\begin{document}

\maketitle

\newpage
\null %%Pour faire une page vide
\newpage

\tableofcontents %%Table des matières
\newpage

\chapter{Sujet}
\chapter{Panification prévisionnelle}
\section{analyse}
\section{méthode}

\chapter{Planification réelle}
\section{Comparaison entre le réel et le prévisionnel}

\chapter{Données quantitatives}
\section{Utilisation d'un gestionnaire de version}
\subsection{Mise en place de la méthode git-flow}
\section{Brain Storming}
%% Retour sur le comparatif, explication des modifications.
%%%git, nb commits / codes ...
%% CMMI MA
%% lien entre III et IV

\chapter{Conclusion}

\end{document}